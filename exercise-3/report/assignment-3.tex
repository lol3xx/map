\documentclass[a4paper,11pt]{article}
\usepackage[utf8]{inputenc}
\usepackage[T1]{fontenc}

\headsep1cm
\parindent0cm
\usepackage{amssymb, amstext, amsmath}
\usepackage{fancyhdr}
\usepackage{lastpage}
\usepackage{graphicx}

\lhead{\textbf{Electronic Market - Jason Implementation}}
\rhead{(Submission: 21.03.2013)}

\cfoot{}
\lfoot{Robert Schmidtke - F121550, Marco Eilers - F121763}
\rfoot{\thepage\ of \pageref{LastPage}}
\pagestyle{fancy}
\renewcommand{\footrulewidth}{0.4pt}

\setlength{\parskip}{4pt}

\begin{document}

\title{Multi-Agent Programming\\Assignment 3: Electronic Market - Jason Implementation}
\author{Robert Schmidtke - F121550, Marco Eilers - F121763}

\maketitle
\newpage

\section{Translation of Design into Implementation}
Since our negotiation algorithm was slightly underspecified in the original design, we reused the algorithm from our JADE implementation: If several partners compete for one product, the seller will always make a new offer to the partner with the lowest previous offer (and the other way round). Both buyers and sellers move twenty percent closer to their respective minimal/maximal price with each offer, until either the difference between two offers becomes less than 10 cent, or their own next offer would be worse than the partner's last offer. When one of these cases is reached, they accept the incoming offer if it is within their price limits and reject it otherwise.

\subsection{Data Structures}
For the traders we had to translate their knowledge about all ongoing negotiations into beliefs. The most important ones are the following:
\begin{itemize}
  \item \texttt{lastPrice(Product,Partner,Price)} The last price offered in the negotiation for
Product with Partner.
  \item \texttt{waitingFor(Product,Partner)} This trader is waiting for a message from Partner 
in the negotiation about Product and may not send messages to others until it
gets a response from Partner. If not waiting, Partner is null.
 \item \texttt{respondTo(Product,List)} List of partners who expect a response concerning 
Product.
\item \texttt{initialSent(Product,Partner)} This trader has sent an initial offer to Partner
concerning Product. If this trader wants to buy Product, it will ignore incoming 
initial offers from Partner concerning Product to avoid having two negotiations 
about the same Product at once.
\item \texttt{sales(Product,List)} List of partners to who want Product. The equivalent for the buyer is \texttt{negotiations(Product,List)}
\item \texttt{offers(Product,MinPrice) } Indicates that this trader wants to sell Product for
at least MinPrice. The equivalent for the buyer is \texttt{requests(Product,MaxPrice)}
\item \texttt{sold(Product)} This Product has been sold, requests concerning it will be
rejected. The equivalent for the buyer is \texttt{bought(Product)}
\end{itemize}

\noindent The backing Java structures (\texttt{ItemDB} and \texttt{ItemDescriptor}) could be reused almost entirely. We only switched from describing items as a set of key-value-pairs of attributes to simply a list of attributes because this could easily be represented in AgentSpeak using arrays. This also reduced the complexity of the configuration files and finding buyers and sellers for matching items in the \texttt{ItemDB}. Since pattern matching using the complete list of attributes for an item is performed, the UUID for each item descriptor was not necessary anymore to uniquely identify an item that is being negotiated.

The \texttt{ItemDB} is the only part of a custom environment, \texttt{ElectronicMarketEnv}. This environment is used only by the matchmaker agent by means of non-internal actions. The matchmaker itself only mitigates between trader agents and the \texttt{ItemDB}.

The previously introduced concept of initializing requests and offers for each trader using configuration files has been realized by defining a common base class for trader agents (\texttt{TraderAgClass}). When an agent's initialization is performed, a configuration file is used (determined by the agent's name) to register offers and requests with the matchmaker. Note that we shipped the same set of of offers and requests as in the previous exercise.



\subsection{Ease of Implementation}


\subsection{Changes to Design}


\section{Notes}

\end{document}
