\documentclass[a4paper,11pt]{article}
\usepackage[utf8]{inputenc}
\usepackage[T1]{fontenc}

\headsep1cm
\parindent0cm
\usepackage{amssymb, amstext, amsmath}
\usepackage{fancyhdr}
\usepackage{lastpage}
\usepackage{graphicx}

\lhead{\textbf{Electronic Market - JADE Implementation}}
\rhead{(Submission: 28.02.2013)}

\cfoot{}
\lfoot{Robert Schmidtke - F121550, Marco Eilers - F121763}
\rfoot{\thepage\ of \pageref{LastPage}}
\pagestyle{fancy}
\renewcommand{\footrulewidth}{0.4pt}

\setlength{\parskip}{4pt}

\begin{document}

\title{Multi-Agent Programming\\Assignment 2: Electronic Market - JADE Implementation}
\author{Robert Schmidtke - F121550, Marco Eilers - F121763}

\maketitle
\newpage

\section{Translation of Design into Implementation}
The design developed using the Prometheus technique has been followed rather closely except some minor changes in the naming of agents and messages. The protocols have been taken from the GAIA methodology, again with minor naming changes. There is exactly one matchmaking agent and multiple trading agents which register their offered and requested items with the matchmaker. Because the ItemDB is held in memory, the matchmaker need not close any open connections to it since it will be removed from memory with the matchmaker's demise.

\subsection{Data Structures}
The \textbf{ItemDB} is a singleton class which consists of two mappings from ItemDescriptors (the said set of attributes that identify an item, implemented as a HashMap) to trading agents: one for requested items and one for offered items. Requests and offers are added as agents enter the market place and they are removed accordingly once a deal has been sealed. Our two agent types are implemented in the classes \textbf{MatchmakerAgent} and \textbf{TraderAgent}. Each TraderAgent has two (possibly empty) \textbf{lists of ItemDescriptors} that hold the requested and offered items which are transmitted (along with the agent ID) during registration with the matchmaking agent. All interactions between agents are implemented as one agent sending a message to the other, who may or may not respond, depending on the kind of the message. MatchmakerAgents have two kinds of behaviours; a \textbf{ContactMatchmakerBehaviour} takes care of (de-)registering offers and requests, whereas a \textbf{MatchmakerBehaviour} does the actual matching between offers and requests. TraderAgents may have any number of \textbf{NegotiationBehaviour}s, depending on the number of currently running negotiations. They control the complete behaviour of all TraderAgents once they have registered with the MatchmakerAgent in the \textbf{setup} method. 

\subsection{Ease of Implementation}
There were several culprits that arose during implementation, the most important one being that the negotiation process has not been specified approriately during the design phase. Several approaches have been explored, among them being the buyer and seller approaching each other in fixed sized steps, but sometimes that resulted in early abortion of negotiation even though the price limits of one party have not been reached yet. It has proven most effective if both seller and buyer approach each other's prices in steps of 25\% of the difference between the current price and their price limit (either maximum price for buyers or minium price for sellers, after which the negotiation would be aborted). 

Another issue was figuring out the item of negotiation: since both parties (buyers and sellers) are allowed to initiate a negotiation, both have a dedicated negotiation Behaviour per item and the actual ItemDescriptor is not transmitted during negotiation, it is necessary to specify a unique identifier for each ItemDescriptor (see below for the UUID).

Concerning the actual implementation, the design was straightforwardly implemented given the Jade framework. This is due to the great flexibility induced by the framework relying on pure Java which has been taken into account during the design phase already. The protocols, messages and events designed were easily realized with the concepts of ACLMessages and Behaviours (see above). We extended the standard Behaviour class in two cases and used a CyclicBehaviour in another case (since TraderAgents always need to respond to requests, if only to reject offers).

\subsection{Changes to Design}
The matchmaking request has been split into two: one for finding buyers for a specific item and one for finding sellers. Consequently the request only contains one ItemDescriptor instead of two lists of ItemDescriptors, specifying the complete set of offers and requests. This implies that for each offer/request a separate matching request is sent to the matchmaking service. This is contrary to the design but has proven to be a more elegant solution in practice. Another extension is that the offered items need not match the requested items exactly, but may as well exceed the specification of the request as long as all attributes of the request are satisfied (a buyer would also buy a green book if he is interested in books in general, without specifying a color).

\section{Notes}
The participants are specified as agents during system startup. They are given paths to initialization files from which they read their requested and offered items. This data is then used for registration with the matchmaking agent. The agents can also be given a time delay which influences the execution time of the setup method. The advantage of a small delay is that the matchmaking agent is definitely available once the agents are fully started (of course, the delay can be set to 0). Furthermore this allows for a more realistic setup of market places where not all participants enter at the same time.

There is a run.sh script (for UNIX and UNIX-like systems) which specifies the matchmaking agent and all trading agents as well as their configuration files and delayed startup times. Sample initialization files are attached. Each trading agent parses ItemDescriptors from these files (which get assigned a type 4 UUID). Except during the registration phase with the matchmaking agent, the ItemDescriptors are not sent during messages, but rather the UUIDs are used for identification of the item that is negotiated over.

The execution of the negotiation process is very verbose: every negotiation initiation, received/accepted/rejected proposal as well as the end of a trade is logged to the console. It is thus possible to track all communication between agents during negotiation. The results of negotiations (successful or not) are printed with the final price. For each additional message (like registering offers and requests), an output line is produced as well.

\end{document}
